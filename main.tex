%%%%%%%%%%%%%%%%%%%%%%%%%%%%%%%%%%%%%%%%%
\documentclass[11pt]{extarticle}
\usepackage[a4paper,left=22mm,right=22mm,top=19mm,bottom=23mm,headsep=13mm]{geometry} 

\usepackage[english]{babel}
\usepackage[utf8x]{inputenc}
\usepackage{amsmath}

\usepackage{graphicx}
\graphicspath{{Images/}}
\usepackage[colorinlistoftodos]{todonotes}
\usepackage{hyperref}
%... set header
\usepackage{lipsum}
\usepackage{graphicx}
\usepackage{fancyhdr}
\usepackage{indentfirst}
\setlength{\parindent}{1.5cm}
\pagestyle{fancy}
\fancyhf{}

\renewcommand{\headrulewidth}{2pt}
\renewcommand{\footrulewidth}{4pt}
\setlength\headheight{110.0pt}
\addtolength{\textheight}{-80.0pt}
\chead{\includegraphics[width=\textwidth]{HeaderLogo.jpg}}
\cfoot{\thepage}

\usepackage{epsfig,graphicx,subfigure,amsthm,amsmath}
\usepackage{color,xcolor}     
\usepackage{xepersian}
\settextfont[Scale=1.2]{XBNiloofar.ttf}
\setlatintextfont[Scale=1]{Times New Roman}

\begin{document}
\begin{titlepage}
\newcommand{\HRule}{\rule{\linewidth}{0.1mm}} 
\centering % Center everything on the page
%---------------------------------------------------------------------------------
%	HEADING SECTIONS 
%---------------------------------------------------------------------------------
\textsc{ }\\[0cm] 
\textsc{\Large به نام خدا  
}\\[0.5cm] % heading course Number
%---------------------------------------------------------------------------------
%	TITLE SECTION (Replace 'TITLE' with the Homework/assignment Name/title)
%---------------------------------------------------------------------------------

\HRule \\[0.4cm]
{
	عنوان:
	\\
	 \huge \bfseries آنالیز سیگنال های الکتروانسفالوگرام حین انجام محاسبات ذهنی
}\\[0.1cm] % Title of your Homework/assignment

\HRule \\[1.5cm]
 گزارش پروژه کارشناسی (2)
 \\
\textsc{\huge دانشکده مهندسی برق
}\\[0.5cm] % heading course name
\textsc{}\\[0.5cm] % heading course name
% Minor heading

%---------------------------------------------------------------------------------
%	AUTHOR SECTION (EDIT THE NAME and T.NO., only)
%---------------------------------------------------------------------------------


\textsc{\Large نگارنده: حامد نجات}
\\[0.5cm]
\textsc{\Large استاد راهنما: دکتر سپیده حاجی پور ساردوئی}
\\
\textsc{\Large استاد ارائه: دکتر مهدی فردمنش}
\\[0.5cm]

%\noalign
%\hline

\emph{}\\ 
\textsc{}
\\[0.5cm]
%\hline


{\large بهار 1400}\\[1cm] % Date, change the \today to a set date if you want to be precise
\includegraphics{CoverLogo.jpg}\\[0.2cm] 
\textsc{\newline}
\textsc{ }\\[0.5cm] 
\vfill % Fill the rest of the page with white-space

\end{titlepage}

% -------------------------------------------------------------------------------------

%\include{GradingRubric}
%\listoftables

% List of Content


\clearpage
\newpage
\paragraph{چکیده}

\paragraph{}
در مغز، فرایند های مختلفی برای انجام یک عملیات ذهنی انجام میشود که برای انجام آنها ارتباطات زمانی-مکانی گوناگونی بین نواحی مختلف در قشر مغز
\lr{(Brain areas)}
 وجود دارد. این ارتباطات را میتوان به صورت یک گراف اتصالات مدل کرد که جزئیات این گراف ممکن است در طول زمان تغییر کنند. یکی از این عملیات های ذهنی، محاسبات عددی ذهنی
\lr{(Mental arithmetic)}
است که در آن، فرد سعی دارد یک عملیات مانند جمع یا ضرب را انجام دهد؛ دو یا چند عدد را در ذهن خود جمع کند و در نهایت حاصل را به دست بیاورد و یا تصمیمی در مورد چند عدد بگیرد؛ مقایسه روی آنها انجام دهد و یا عددی را بر اساس معیاری مانند اول بودن انتخاب کند. در این پروژه سعی می شود به کمک ابزار های رابط مغز-رایانه
\lr{(BCI)}
 که عموما از داده های الکتروانسفالوگرام
\lr{(EEG)}
 جمع آوری شده حین انجام این عملیات ها هستند، ویژگی هایی مانند شکل زمانی یا طیف انرژی فرکانسی استخراج شود 
 \lr{(Feature extraction)}
 و به کمک ابزار هایی مانند تحلیل همبستگی ، طبقه بندی 
 \lr{(Classification)}
 ، تخمین منابع سیگنال، آنالیز روابط علی 
 \lr{(Causality)}
  و خوشه بندی سیگنال ها
 \lr{(Clustering)}
  گراف اتصالات
 \lr{(Connectivity graph)}
  این نواحی و یا حسگر های الکتروانسفالوگرام در حین انجام این عملیات ها تخمین زده شود. در نهایت خواهیم دید که ویژگی های نهانی وجود دارند که بر اساس آنها میتوان این سیگنال ها را بر اساس کیفیت و سرعت محاسبه با دقت بسیار بالایی (حدود 90 درصد) تشخیص داد.
 
 \paragraph{واژگان کلیدی:}
 \paragraph{}
 رابط مغز-رایانه
 \lr{(BCI)}
 ؛ سیگنال الکترواسنفالوگرام
 \lr{(EEG signal)}
 ؛ محاسبات ذهنی
 \lr{(Mental arithmetic)}
 \\
 ؛ خوشه بندی
 \lr{(Clustering)}
 ؛ طبقه بندی
 \lr{(Classification)}
 ؛ گراف اتصالات
 \lr{(Connectivity analysis)}
 \\
 ؛ نواحی مغزی
 \lr{(Brain areas)}
 ؛ استخراج ویژگی
 \lr{(Feature extraction)}
 ؛ آنالیز طیفی
 \lr{(Spectral analysis)}
 \\
 ؛ آنالیز علیت
 \lr{(Causality analysis)}
 ؛ آنالیز اطلاعات مشترک
 \lr{(Mutual information analysis)}


\clearpage
\newpage

\tableofcontents % Contents

\clearpage
\newpage

\section{پیشگفتار}
\paragraph{}
در این بخش، نگاهی کوتاه به جنبه های گوناگون در مباحث مرتبط با موضوع ارتباط مغز و رایانه میپردازیم، سپس مفاهیمی کلی در این زمینه ها را به طور خلاصه بررسی میکنیم.
\subsection{ارتباط مغز-رایانه}
ارتباط مغز و رایانه بیشتر به هدف کمک به بیماران مغز و اعصاب و حل مشکلات آنها به وجود آمد، اما در اصل شناخت عملکرد و ساختار مغز نیز به عنوان هدف اصلی آن مورد توجه قرار گرفت. برای ارتباط مغز با رایانه ابزار های گوناگونی به جهت ثبت سیگنال های مغزی، پردازش آن، طبقه بندی آن و حتی ارسال فیدبک آن به مغز طراحی شده اند. در فصل سیگنال های مغزی بیشتر به این موضوع خواهیم پرداخت.

\begin{figure}[h!]
	\centering
	\includegraphics[width=16cm]{Images/5.png}
	\caption{شکل کلی یک سیستم مبتنی بر ارتباط مغز-رایانه}
	\label{fig:1}
\end{figure}

\subsection{محاسبات ذهنی}
مغز بسیار شبیه یک کامپیوتر عمل میکند؛ مجموعه زیادی از ورودی هارا از طریق پایانه های حسی مانند بینایی و شنوایی میگیرد و با انجام پردازش های گوناگون بر روی آنها، خروجی ای مانند تصمیم گیری یا گفتن یک جمله می دهد. بخش زیادی از این پردازش ها در درون خود محاسبه دارند؛ 
فرایند های محاسباتی به صورت جامع شامل بخش های زیادی از مغز میشوند. برای مثال هنگامی که یک آشپز غذا میپزد مواد لازم را می شمارد و یا یک دارو ساز مقادیر مختلفی از دارو ها را جمع می زند. مثال های بسیار بیشتری از محاسبات ذهنی نیز وجود دارند، می توان گفت که به ندرت فرایند های فکری در خود محاسبه ای ندارند.
\subsection{الکتروانسفالوگرام}
الکتروانسفالوگرام
\footnote{EEG}
سیگنالی الکتریکی است که از روی پوست سر ثبت می شود. در فصل سیگنال های مغزی به این خواهیم پرداخت که چرا برای این پروژه، این دسته سیگنالها فعلا مناسب هستند. رزولوشن این سیگنالها معمولا در حدود 100-500 هرتز بوده و بین 20-60 کانال نیز دارند. ثبت آنها از اکثر روش های دیگر ثبت سیگنال مغزی هزینه و زمان کمتری می برد.
\subsection{ویژگی ها}
در هر سیگنال و داده ای، ویژگی های
\footnote{Feature}
 زیادی چه به صورت نهفته 
\footnote{Latent}
و چه به صورت عادی موجود هستند. بررسی این ویژگی ها بسیار مهم است؛ چرا که عملا تنها روش بررسی ارتباط بین چند فرایند و یا دسته بندی آنها بررسی همین ویژگی ها است. برای مثال ممکن است اتفاق خاصی در یک ناحیه مغز، منجر به تولید سیگنالی با شکل زمانی خاص شود، بررسی وجود این ویژگی در بقیه کانال ها می تواند میزان انتشار این سیگنال، همبستگی نواحی و یا عملکرد یک ناحیه هنگام یک فعالیت خاص را بیان کند. از آنجا که بیشتر فعالیت های مغزی به صورت زمانی سیگنال های خاصی تولید می کنند، ویژگی هایی مانند طیف فرکانسی و شکل زمانی بیشترین استفاده را در تحلیل سیگنال های مغزی دازند. به دلیل اهمیت فرکانس در سیگنال های الکتروانسفالوگرام نیز دسته بندی ای بر روی فرکانس های مختلف آن و کارکرد کلی مربوط به آن انجام شده است که در شکل زیر آورده شده است:
\begin{figure}[h!]
	\centering
	\includegraphics[width=16cm]{Images/1.png}
	\caption{کارکرد های کلی مغز در فرکانس های مختلف سیگنال های الکترانسفالوگرام}
	\label{fig:2}
\end{figure}

\subsection{تحلیل اتصالات}
در یک سیستم پیچیده همانند مغز، از لحاظ الکترومغناطیسی مقدار بسیار زیادی منبع سیگنال مختلف وجود دارد؛ حتی میتوان یک نورون را به صورت چند منبع مختلف مدل کرد. اگر مغز را به صورت کلی تری بخش بندی کنیم؛ مثلا نواحی ای در حدود اندازه چند سانتیمتر در چند سانتیمتر را یک بخش مجزا در نظر بگیریم، میتوان سیگنال ثبت شده از نزدیک ترین الکترود به آن ناحیه را به عنوان مشخصه آن ناحیه در نظر گرفت. حال متغیر هایی مانند فرکانس، مقدار میانگین و شکل زمانی میتوانند بین چند ناحیه مشترک باشند و یا اصلا مربوط نباشند، این متغیر ها حتی ممکن است در طول زمان تغییر کنند. بر اساس میزان شباهت این متغیر ها و یا متغیر های پیچیده تر (ویژگی های سطح بالا تر مانند آماره های مرتبه بالا) می توان معیاری درباره متصل بودن یا نبودن دو ناحیه از دید عملکرد آنها بیان کرد. در فصل مدل های گراف اتصالات، بیشتر به این موضوع خواهیم پرداخت.
\subsection{طبقه بندی}
از دید علم داده
\footnote{\lr{Data science}}
اگر بتوان یک دسته از داده ها را با دقت بالایی بر اساس برچسپ گذاری 
\footnote{Labeling}
خاصی طبقه بندی
\footnote{Classification}
 کرد، میتوان به اعتبار آن برچسپ گذاری اعتماد بیشتری کرد. حال اگر گراف اتصالات به دست آمده بر روی مغز درست باشد، با اجرای یک طبقه بندی بر روی این دسته داده ها بر اساس گراف به دست آمده باید بتوان میزان صحت بالایی به دست آورد. اگر چه که به طور کلی هدف از طبقه بندی این نیست، اما در این پروژه یکی از کاربرد های طبقه بند ها همین اعتبار سنجی است. 
\subsection{خوشه بندی}
به کمک خوشه بندی 
\footnote{Clustering}
می توان بر اساس ویژگی های نهفته سیگنال ها را دسته بندی کرد، مزیت خوشه بندی عدم نیاز به برچسپ برای داده ها است که زمانی که اطلاعات پیشین
\footnote{\lr{Prior information}}
در مورد داده ها نداریم، بسیار ابزار مفیدی خواهد بود. خوشه بندی همچنین میتواند میزان شباهت دو سیگنال مختلف را به صورت نسبی بدهد؛ برای مثال در حالتی که میان 20 سیگنال مختلف دو سیگنال خاص در یک دسته نزدیک قرار بگیرند نشان از ارتباط بیشتر آن دو در سیستمی شامل همان 20 سیگنال دارد در حالی که ممکن است از نظر مشاهده کلی همه سیگنال ها شبیه هم باشند. در فصل خوشه بندی بیشتر در این باره خواهیم پرداخت.

\clearpage
\newpage
\section{سیگنال های مغزی}
به طور کلی سیگنال های مغزی را می توان به دو دسته تصویر و سیگنال الکترومغناطیسی تقسیم کرد؛ هر چند ماهیت سیگنال تصویری نیز ممکن است الکترومغناطیسی باشد. هدف از این فصل، آشنایی کلی با سیگنال ها و اطلاعات ثبت شده از مغز بوده و در نهایت علت استفاده از سیگنال های انسفالوگرام در این پروژه را بررسی خواهیم کرد.

\subsection{تصویر برداری مغزی}
تکنولوژی های بسیاری برای تصویر برداری از اجسام، به صورت دو یا سه بعدی وجود دارد. در این زیر بخش مهمترین تکنولوژی ها در بحث تصویر برداری مغز را به صورت خلاصه بررسی میکنیم.
\subsubsection{MRI}
تصویر برداری بر مبنای تشدید مغناطیسی 
\footnote{\lr{Magnetic Resonance Imaging}}
رایج ترین روش تصویر برداری از مغز به هدف بررسی های پزشکی است. در این روش، تصویری تک زمان از مغز به صورت سه بعدی برداشته می شود که بر مبنای تشدید مولکول های قطبی در مغز است. به دلیل فراوانی مولکول آب در بدن انسان و قطبیت بسیار بالای آن نسبت به اکثر مواد دیگر، تقریبا همه این تشدید مغناطیسی در بدن مربوط به مولکول های آب بوده و برای جلوگیری از تداخل های بی شمار مغناطیسی نواحی مختلف بدن، پاسخ مغناطیسی به صورت متمرکز از هر ریز حجم 
\footnote{Voxel}
جداگانه ثبت می شود. این ریز حجم در واقع یک مکعب کوچک بوده که اندازه آن دقت مکانی
\footnote{Resolution}
سیستم تصویر برداری است؛ تقریبا مکعبی با اضلاع در حدود 1 تا 10 میلیمتر که بر اساس پاسخ آن به تحریک تک فرکانس مغناطیسی، میزانی از سفت یا نرم بودن بافت خود می دهد.
از آنجا که در بافت های نرم تر، تشدید ساده تر اتفاق می افتد پاسخ بافت های مختلف متفاوت می شود. از این روش می توان به ساختار کلی مغز پی برد، بیماری های فیزیولوژیکی مانند تومور ها را تشخیص داد و یا آسیب های فیزیکی را بررسی نمود.

\begin{figure}[h!]
	\centering
	\includegraphics[width=12cm]{Images/1.jpg}
	\caption{نمونه تصویر تشدید مغناطیسی}
	\label{fig:3}
\end{figure}

\subsubsection{fMRI}
اگرچه تصاویر 
MRI
اطلاعات بسیار با ارزشی در مورد ساختار مغز در اختیار ما قرار می دهند، اما در مورد تغییرات زمانی هیچ اطلاعاتی ندارند. علاوه بر این مشکل، حتی اگر تصاویر 
MRI
به صورت زمانی ثبت گردند، چندان کاربردی ندارند چرا که ساختار فیزیکی و معماری مغز در طول زمان تغییر خاصی نمیکند؛ چیزی که در طول زمان اهمیت دارد فعالیت مغزی است. برای حل این مساله، تصویر برداری کارکردی تشدید مغناطیسی 
\footnote{\lr{Functional Magnetic Resonance Imaging}}
به وجود آمد. برای این روش، به جای پاسخ تشدید کلی هر ریز حجم در لحظه، از پاسخ زمانی فعالیت هر ریز حجم در طول زمان استفاده می شود. این پاسخ زمانی با هموگلوبین خون در رگ های مغزی هر ناحیه رابطه دارند؛ دلیل این پدیده وجود آهن در هموگلوبین خون و تشدید مغناطیسی شدید تر آن است. به این معیار، وابستگی به میزان اکسیژن خون 
\footnote{\lr{Blood Oxygen Level Dependant (BOLD)}}
نیز گفته می شود.
 وجود بیشتر اکسیژن خون باعث بیشتر شدن هموگلوبین و نشانه فعالیت بیشتر آن بخش در آن زمان است. دقت مکانی در این سیستم از 
MRI
کمی کمتر است اما در عوض، سیگنال تصویر اطلاعات زمانی دارد و به کمک آن می توان عملکرد های مغزی را بررسی نمود. در نهایت، خروجی این سیستم یک سیگنال 4 بعدی است.

\begin{figure}[h!]
	\centering
	\includegraphics[width=14cm]{Images/2.jpg}
	\caption{نمونه تصویر کارکردی تشدید مغناطیسی؛ بخش های سفید تر نشانه فعالیت بیشتر هستند.}
	\label{fig:4}
\end{figure}

\subsubsection{SPECT/PET}
بخش دیگری از روش های رایج تصویر برداری مغزی، روش های مبتنی بر پرتوزایی
\footnote{Radioactivity}
 است. این روش ها نیمه تهاجمی بوده و نسبت به روش های تشدید مغناطیسی برای بیمار یا فرد مورد تصویر برداری سخت تر هستند. در بحث ارتباط مغز-رایانه کمتر از سیگنال ها و تصاویر این سیستم ها استفاده می شود؛ چرا که هم هزینه ی این تصویر برداری بیشتر بوده واز لحاظ پزشکی ممکن است عوارضی برای فردی که مورد تصویر برداری است داشته باشد، هم برای بسیاری کاربرد ها تصاویر تشدید مغناطیسی کافی هستند. 
\paragraph{}
کلیت این روش ها، استفاده از ماده های حاجب یا آشکار گر 
\footnote{Contrast}
و اندازه گیری میزان تشعشع از بافت ها است. طبیعتا این ماده آشکارگر باید به طریق خارجی به بدن وارد شود؛ در نتیجه به بیمار این آشکارگر ها که خاصیت پرتوزایی دارند تزریق می شود و دستگاه میزان این پرتوزایی را بعد از مدتی از بافت های مربوطه دریافت میکند. این روش، با رزولوشن مکانی مشابه MRI جزئیات ساختاری بافت ها را می دهد؛ همچنین میتوان در طول زمان حرکت این آشکار ساز را ثبت کرد. حتی جهت بررسی تومور ها یا بافت خاص، ماده ای که آن بافت خاص بیشتر جذب می کند را تبدیل به ماده پرتوزا کرده و به بدن فرد تزریق میکنند؛ در بعضی کاربرد ها حین تصویربرداری مغزی پرتوزا به دلیل جذب زیاد تر گلوکوز توسط مغز، گلوکوز پرتوزا را به بدن تزریق میکنند.

\begin{figure}[h!]
	\centering
	\includegraphics[width=16cm]{Images/2.png}
	\caption{نمونه تصویر مبتنی بر پرتوزایی}
	\label{fig:5}
\end{figure}

\subsection{سیگنال های الکترومغناطیسی}
بخش عمده ای از کاربرد های مغز-رایانه، بر اساس سیگنال های زمانی الکترومغناطیسی است که توسط الکترود های الکتریکی یا مغناطیسی ثبت می شوند. ثبت این سیگنال ها توسط الکترود ها انجام می شود که دقت زمانی بسیار بالاتری نسبت به روش کارکردی 
fMRI
و یا 
SPECT
دارند. علاوه بر این مزیت اصلی، این دسته سیگنال ها ثبت ساده تری نسبت به روش های تصویر برداری دارند که یا سیستم بسیار پیچیده و زمان بری مثل دستگاه 
MRI
دارند و یا نیاز به تزریق ماده پرتوزا مانند 
PET
دارند.

\subsubsection{ٍالکتروکورتیکوگرام}
ماهیت سیگنال های مغزی به طور کلی الکتروشیمیایی است و واسطه ها برای انتقال آن، یون ها و مولکول های نوروترنسمیتر
\footnote{\lr{Neurotransmitter}}
هستند. فعالیت های مغزی باعث ایجاد پتانسیل های الکتریکی متفاوتی در طول زمان در  همه جای مغز می شوند؛ اگر یک الکترود بر هر جای مغز بگذاریم، پتانسیل های الکتریکی متفاوتی در طول زمان مشاهده میکنیم. الکترود های سطحی یا الکترود های تماسی، میتوانند با جراحی بر سطح قشر مغز و در درون جمجمه متصل شوند و سیگنالی با دقت زمانی بسیار بالا در حدود 500 هرتز و خطای ناچیز از پتانسیل تعداد بسیار کمی نورون در کنار هم بدهند. این الکترود ها معمولا به صورت آرایه های جدولی و منظم که دارای تعداد بسیار زیادی الکترود نزدیک به هم هستند به بافت قشر مغز متصل می شوند، در موارد خاصی نیز الکترود ها به درون قشر فرو می شوند و به دلیل ورود به کورتکس، به آن الکتروکورتیکوگرام 
\footnote{\lr{Eletro-Corticogram(ECoG)}}
گفته می شود.
مسلما این روش یک روش تهاجمی است؛ اما به دلیل مزیت های دقت زمانی-مکانی بسیار بالای این روش، بسیاری از آزمایش های علوم اعصاب (عمدتا بر روی حیوانات) به این روش انجام شده اند. اگر چه که این روش سختی های زیادی دارد و بسیار کمتر بر روی انسان انجام می شود اما موارد انجام آن بر روی انسان کم نبوده و برای تحقیقاتی بر روی بیماران صرعی، افراد دارای آسیب های مغزی و یا بیمارانی که توده های مغزی دارند انجام شده است.

\begin{figure}[h!]
	\centering
	\includegraphics[width=10cm]{Images/4.png}
	\caption{نمونه فرضی الکتروکورتیکوگرام}
	\label{fig:6}
\end{figure}

\subsubsection{ٍٍالکتروانسفالوگرام}
همانطور که در بخش قبلی توضیح داده شد، استفاده از الکتروکورتیگوگرام روشی تهاجمی است و برای ثبت سیگنال حین انجام بسیاری از فعالیت ها مانند ثبت سیگنال حین ورزش نمیتوان به سادگی از چنین روش تهاجمی ای استفاده کرد. حال اگر الکترود های ثبت پتانسیل را بر روی پوست سر قرار دهیم، همچنان پتانسیل الکتریکی ای میتوان ثبت کرد که هر چند به دقت مکانی روش های تهاجمی نیست و نویز بیشتری به علت فاصله مکانی با نورون های قشر مغز در آن مشاهده می شود، اما همچنان بسیاری از کاربرد های ارتباط مغز-رایانه با این سیگنال که به دلیل ثبت آن از روی جمجمه، الکتروانسفالوگرام
\footnote{\lr{ٍElectro-Encephalogram(EEG)}}
 گفته میشود قابل انجام است. 

\begin{figure}[h!]
	\centering
	\includegraphics[width=17cm]{Images/3.JPG}
	\caption{سیستم 10-20 الکتروانسفالوگرام}
	\label{fig:7}
\end{figure}

\paragraph{}
سیستم های ثبت الکتروانسفالوگرام چندین استاندارد مختلف دارند. در بسیاری از کاربرد ها، از سیستم 10-20 استفاده می شود که طبق شکل بالا الکترود ها روی سطح سر قرار می گیرند. سعی می شود الکترود ها رسانایی بالایی با سطح پوست سر داشته باشند تا اثر امپدانس های سنسور کمتر دیده شود، به این منظور به محل تماس الکترود ها ژل رسانای خاصی آغشته میکنند. در نهایت، مقادیر ولتاژ ثبت شده نسبت به یک الکترود مرجع و یا میانگین پتانسیل الکترود ها نسبت به مرجعی در خودشان و در موارد خاصی پتانسیل الکترود ها نسبت به یکدیگر، ثبت می شوند. دستگاه ثبت این سیگنال به صورت یک کلاه بوده به راحتی قابل حمل است. در میان روش های کاربردی ثبت سیگنال مغزی، نسبت به هر روش دیگری انسفالوگرام ها روش ساده تری هستند و بعد از مزیت دقت زمانی آنها، مزیت سادگی ثبت آنها دلیل استفاده از آنها است؛ چرا که حتی میتوان فعالیت مغزی یک فرد را در کار های روز مره یا ورزش نیز به راحتی ثبت کرد. به نسبت سایر روش ها، داده های بیشتری از این دسته سیگنال ها نیز موجود است.

\begin{figure}[h!]
	\centering
	\includegraphics[width=16cm]{Images/4.JPG}
	\caption{یک کلاه ثبت سیگنال الکتروانسفالوگرام به همراه نمونه سیگنال ثبت شده}
	\label{fig:8}
\end{figure}

با توجه به این که مختصات مکانی الکترود ها و سیگنال ثبتی از آنها مشخص است و همچنین عامل ایجاد پتانسیل های الکتریکی را می توان به صورت یک دوقطبی الکتریکی در نظر گرفت، به روش های جبری میتوان تخمینی از منبع اصلی یک حالت خاص از سیگنال را تخمین زد. روش های تخمین منبع و مکان یابی منابع مغزی در مقالات سال های اخیر روش های متنوعی برای مکان یابی این منابع توسعه داده و پیاده سازی کرده اند که در فصل مدل های گراف اتصالات به آن اشاره خواهیم کرد.

یک دسته از مشکلات مهم در سیگنال های انسفالوگرام، نویز ها
\footnote{\lr{Noises}}
 و آرتیفکت ها 
\footnote{\lr{Artifacts}}
 هستند. همانطور که گفته شد، این سیگنال از روی پوست سر ثبت می شود که میتواند برای هر شخص ویژگی های متفاوتی با بقیه داشته باشد؛ مثلا تعریق و یا خشکی پوست ممکن است در طول ثبت سیگنال تغییراتی بر سیگنال اصلی ایجاد کنند. فاصله زیاد الکترود از نورون های  درون قشری نیز باعث می شود دامنه سیگنال آنها بسیار کمتر دیده شود؛ همچنین پالس های ضربان قلب، سیگنال های عضلانی مختلف نیز به نسبت سیگنال اصلی ممکن است دامنه بالایی داشته باشند و در برخی و یا همه ی الکترود های مغزی دیده شوند. مهمترین آرتیفکت در انسفالوگرام، آرتیفکت سیگنال عضلانی پلک یا الکتروآکلوگرام
\footnote{\lr{Electro-Oclulogram(EOG)}}
 است. این سیگنال، از حرکات عضلات چشمی و پلک زدن ایجاد می شود و پتانسیل قابل توجهی را به خصوص در الکترود های جلوی سر ایجاد میکند. در بسیاری موارد، توسط الکترود هایی این سیگنال نیز ثبت می شود تا در پیش پردازش بتوان آنرا مستقیما از سیگنال اصلی جدا کرد؛ چرا که حرکات چشم سیگنال های مختلفی ممکن است ایجاد کنند که اصلا قابل تشخیص نباشد. همچنین برای سیگنال قلبی
\footnote{\lr{Electro-Cardiogram(ECG or EKG in German)}}
 نیز همین رویکرد انجام می شود؛ اگر چه که سیگنال قلبی منظم تر از سیگنال چشمی است و تشخیص آن به نسبت راحت تر است اما الکتروکاردیوگرام نیز همزمان با ثبت الکتروانسفالوگرام ثبت می شود تا در پیش پردازش اثر آن نیز از سیگنال اصلی کاسته شود. آرتیفکت های رایج دیگر، حرکات سر و دست، تنفس، صحبت کردن و حرکات دیگر فک هستند.
 
 نویز ها نیز دسته دیگری از اثرات نا مطلوب هستند. نویز میتواند منابع بسیار متنوعی داشته باشد؛ معروف ترین آنها نویز اثرات برق شهر است که معمولا در هر سیگنالی وجود دارد؛ به دلیل تک فرکانس بودن برق شهر حذف آن از اکثر نویز های دیگر ساده تر است. سایر نویز ها میتوانند عامل های محیطی و حتی درونی ناشناخته داشته باشند که معمولا به روش های دیگری مانند میانگین گیری و تکرار آزمایش حذف می شوند.

\subsubsection{مگنتوانسفالوگرام}
طبق فیزیک الکترومغناطیس، هر جریان الکتریکی یک میدان مغناطیسی ایجاد می کند. منابع عامل سیگنال های الکتروانسفالوگرام نیز در اصل دو قطبی های منبع جریان هستند؛ در نتیجه سیگنال مغناطیسی ای نیز باید ایجاد کنند. به این سیگنال، مگنتوانسفالوگرام 
\footnote{\lr{Magneto-Encephalogram(MEG)}}
گفته می شود که سیستم های ایتاندارد آن مشابه الکتروانسفالوگرام هستند؛ سیگنال به صورت چند کاناله از الکترود های مغناطیسی دور مغز ثبت می شود. تفاوت عملی آن در دستگاه ثبت آن است که همانند دستگاه 
MRI
 نسبتا بزرگ و پیچیده است. به دلیل هزینه بیشتر و سختی ثبت، این سیگنال ها نسبتا کمتر از الکتروانسفالوگرام استفاده می شوند.
 
\begin{figure}[h!]
	\centering
 	\includegraphics[width=10cm]{Images/3.png}
 %	\caption{اساس ایجاد سیگنال مگنتوانسفالوگرام}
%\end{figure}
 
%\begin{figure}[h!]
%	\centering
	\includegraphics[width=10cm]{Images/5.jpg}
	\caption{اساس فیزیولوژیکی و دستگاه ثبت سیگنال مگنتوانسفالوگرام}
	\label{fig:9}
\end{figure}
 

\clearpage
\newpage
\section{مدل های گراف اتصالات مغزی و منابع سیگنال ها}
در این بخش، به اساس مدل های گرافی سیستم ها و نحوه تخمین و مدلسازی یک سیستم بر اساس سیگنال ای ثبت شده از پایانه های آن می پردازیم. در آخر نیز بحث جداسازی و تخمین منابع را بررسی میکنیم.

\subsection{اتصالات منابع بر مبنای مدل}

دسته بزرگتری از آنالیز اتصالات چند منبع، آنالیز مبتنی بر مدل
\footnote{\lr{Model-based connectivity analysis}}
است. در این زیر بخش، ابتدا مدلسازی مبتنی بر پویایی و علّیت و سپس دو حالت دیگر این آنالیز را یعنی آنالیز مبتنی بر همبستگی یا انسجام
\footnote{\lr{Correlation/coherence analysis}}
و آنالیز مبتنی بر ویژگی مشترک
\footnote{\lr{Common feature analysis}}
را بررسی میکنیم. 

\subsubsection{مدلسازی علّی پویا}

\begin{figure}[hb!]
	\centering
	\includegraphics[width=10cm]{Images/6.jpg}
	\caption{\lr{Dynamic causal modeling (DCM)}}
	\label{fig:10}
\end{figure}

اگر منبع یک سری سیگنال را با یک تابع پویا
\footnote{\lr{Dynamic function}}
تخمین بزنیم، این تابع یک سری پارامتر حالت خواهد داشت. این تابع در واقع یک مدل فضا-حالت است که به ازای یک حالت، میتوان در طول زمان آنرا پیش بینی کرد. برای هر منبع تخمین زده شده که در اینجا، کانال های الکتروانسفالوگرام هستند؛ یک تابع به صورت زیر به ازای هر کانال تخمین زده می شود؛ که در آن مقدار زمان و مقادیر گره ها با یک دسته ضریب (یالهای گراف) به هم مرتبط اند:
$$ f_i(t, x_1, x_2, ..., x_n) = S_i(t)$$
حال با مقایسه توابع و پارامتر های آنها برای هر دو کانال، میتوان ارتباط آنها از نظر زمانی یا مکانی را بررسی کرد. به خاطر این ویژگی ها، به این روش مدلسازی علّی پویا
\footnote{\lr{Dynamic causal modeling}}
 گفته می شود. برای مثال، اگر دو تابع به صورت زیر ارتباط داشته باشند:
$$ f_i(t) = \alpha f_j(t-a)$$
میتوان گفت که گره i در این گراف اطلاعات گره j را با تاخیر a و تضعیف با تقویت 
$\alpha$
منتشر می کند، پس به صورت جهت دار به آن متصل است و نسبت به آن معلولیت دارد. البته دقت شود که در مدل های واقعی، این ارتباط به این صورت ساده تاخیر و تقویت یا تضعیف نیست، بلکه پارامتر های حالت یا زمان صرفا ممکن است شباهت داشته باشند و همچنین تعداد گره ها و یالهای گراف مدل بسیار زیاد ممکن است باشد. اما معمولا اگر ارتباطی در سیستم ها به صورت علّی یا خطی باشد، به این روش با کمی تقریب به دست خواهد آمد. نحوه مدل سازی و تخمین البته بسته به پارامتر های سیستم و میزان پیچیدگی آن ممکن است فرق کند، در مثال تصویر یک حالت خاص از این روش برای گرافی با دو گره را میبینیم.

\subsubsection{اتصالات بر مبنای معیار علیت گرنجر\lr{(Granger causality)}}

این معیار به زبان ساده، این را مورد آزمون قرار میدهد که آیا بین دو سری زمانی X, Y ، مدلی که با آن Y را از روی نمونه های قبلی خودش تخمین بزنیم خطای کمتری خواهیم داشت و یا بر اساس X و خود آن بسازیم خطای کمتری خواهیم داشت. فرض تهی (\lr{Null hypothesis}) این است که اضافه کردن X به تخمین گر تاثیری ندارد و فرض جایگزین (\lr{Alternative hypothesis}) نیز این است که سری زمانی X به تخمین Y کمک می کند. در شکل زیر، بیان این روش را میبینیم: 

\begin{figure}[h!]
	\centering
	\includegraphics[width=15cm]{Images/12.jpg}
	\caption{\lr{Granger causality test}}
	\label{fig:11}
\end{figure}

\subsubsection{اتصالات منابع بر مبنای ویژگی ها}

در این روش ها، در نظر گرفته می شود که در یکی از منابع خاص، به ازای یک فعالیت یک ویژگی خاصی ایجاد می شود. حال با فرض وجود یک شبکه گرافی در این سیستم، این ویژگی بسته به میزان متصل بودن در سایر منابع نیز باید دیده شود. بدین ترتیب، اگر ابزاری تعریف گردد که میزان شباهت دو سیگنال در یک بازه ثابت را بیان کند میتوان این فرض را نمود که این میزان بیانگر متصل بودن و اندازه این اتصال بین دو منبع یا پایانه خواهد بود. 

در برخی روش های بررسی اتصال، از معیار های همبستگی و انسجام استفاده می شود. این معیار ها، به طور کلی سه دسته هستند. هر کدام را جداگانه به همراه مزیت ها و معایب آنها بررسی میکنیم:

\begin{itemize}
	\item[1]
	اولین معیار، همبستگی خطی پیرسون
	\footnote{\lr{Pearson correlation}}
	 است که رایج ترین نوع همبستگی بوده و برای دو سری بردار با سیگنال به صورت زیر تعریف می شود:
	$$ \rho_{x, y} = \frac{\Sigma(x[n] - \bar{x[n]})(y[n] - \bar{y[n]})}{\sqrt{\Sigma(x[n] - \bar{x[n]})^2}\sqrt{\Sigma(y[n] - \bar{y[n]})^2}} $$
	ویژگی مثبت این همبستگی، سادگی محاسبه آن و پیچیدگی کمتر محاسباتی برای استخراج شباهت های آماری پایه مانند شباهت شکل موج کلی است. اما ایراد اصلی آن، حذف بعد زمان از معیار است که به دلیل میانگین گیری در طول زمان اتفاق می افتد. همچنین، پدیده هایی مانند تاخیر در سیگنال ها و یا جابجایی سیگنال در آن در نظر گرفته نمی شود و عملا ممکن است تاخیر حتی به میزان کم، میزان همبستگی را به شدت تغییر دهد و یا جابجایی زمانی بین دو سیگنال خاص اصلا تغییری در همبستگی نداشته باشد. با این تبدیل به تنهایی نیز نمیتوان روابط جهت دار در گراف مانند علّیت را استخراج کرد.
	\item[2]
	معیار رایج دیگر که در مباحث سیگنال ها بیشتر استفاده می شود، همبستگی طیفی 
	\footnote{\lr{Spectral coherence}}
	است. این معیار به صورت زیر تعریف می شود:
	$$ C_{x, y} (w) = \frac{|S_{xy}(w)|}{\sqrt{S_{xx}(w)}\sqrt{S_{yy}(w)}}, S_{xy}(w) = DTF(R_{xy}[n]), R_{xy}[n] = \sum_{m = -\infty}^{+\infty} x^{*}[m]y[m+n] $$
	که در آن R همبستگی مشترک دو سیگنال و S طیف های تابع همبستگی مشترک دو سیگنال است که معمولا از روش فوریه گسسته به دست می آید. اولین و مهمترین ویژگی این معیار، دخالت زمان و تاخیر در آن است. در صورتی که دو سیگنال از لحاظ طیفی شباهت داشته باشند، این معیار به حداکثر خود نزدیک می شود. بر اساس قضایای سیگنال و سیستم نیز در دو سیگنال ارگودیک در صورتی که طیف ها یکپارچه باشند، علیت بین دو سیگنال وجود دارد. البته جهت این علیت با این معیار تعیین نمی شود و نیازمند بررسی فاز و تاخیر است؛ اما حداقل میتوان اثر زمان را بهتر بررسی نمود.
	\item[3]
	تابع طیف ها مقداری مختلط دارد و برای همین در بخش قبلی درون قدر مطلق قرار گرفته است. اگر این قدر مطلق را برداریم؛ مقداری مختلط به دست می آید که در واقع معیار دیگر همبستگی طیفی است؛ با این تفاوت که مقدار آن مختلط است. این معبار همبستگی تفاظل فاز 
	\footnote{\lr{Phase difference}}
	ضعف معیار قبلی یعنی عدم تاثیر تقدم و تاخر را رفع می کند: 
		$$ C_{x, y} (w) = \frac{S_{xy}(w)}{\sqrt{S_{xx}(w)}\sqrt{S_{yy}(w)}}, S_{xy}(w) = DTF(R_{xy}[n]), R_{xy}[n] = \sum_{m = -\infty}^{+\infty} x^{*}[m]y[m+n] $$
	همانطور که گفته شد؛ این معیار میتواند (در صورتی که سیگنال ها بد تعریف نباشند) میزانی از جهت علیت هارا بیان کند. معمولا سیگنال ها در طول زمان به صورت قطعه قطعه یا پنجره ای در محاسبه همبستگی ها استفاده می شوند تا تغییرات زمانی در حدود ثانیه در ساختار گراف خروجی در نظر گرفته شود.
\end{itemize}

\subsection{اتصالات منابع بدون مدل}

روش های بدون مدل 
\footnote{\lr{Model-free connectivity analysis}}
که بیشتر در تئوری اطلاعات تعریف و استفاده میشوند؛ از دیگر روش های استخراج گراف اتصالات بین چند سیگنال در یک سیستم هستند. این روش ها به خطی بودن یا ارگودیک بودن دیگر وابسته نیستند و از فرض های آماری استفاده می کنند؛ یکی از مثال های این روش ها روش اطلاعات متقابل 
\footnote{\lr{Mutual information}}
است که در آن، سیگنال ها به صورت فرایند های تصادفی در نظر گرفته شده و وابستگی های کلی بین دو فرایند استخراج می شود؛ این روش نیز مانند همبستگی پیرسون متغیر زمان را در خود ندارد و در محاسبات آن بعد زمان حذف می شود. 

\begin{figure}[h!]
	\centering
	\includegraphics[width=12cm]{Images/10.jpg}
	\caption{\lr{Discrete definition}}
	\label{fig:12}
\end{figure}

\begin{figure}[h!]
	\centering
	\includegraphics[width=12cm]{Images/11.jpg}
	\caption{\lr{Continuous definition}}
	\label{fig:13}
\end{figure}

روش دیگر که پیچیده تر است نیز روش آنتروپی انتقال است که در آن، میزان کاهش عدم قطعیت در صورت داشتن مقادیر یکی از سیگنال ها نسبت به بقیه و برعکس بررسی می شود. به دلیل اینکه این روش ها نیاز به داده های زیادتر و محاسبات بسیار بیشتری دارند؛ از آنها در این پروژه استفاده نمی شود و حتی در کتابخانه های علوم اعصاب محاسباتی نیز تا کنون پیاده سازی ای از آنها جهت استفاده بر روی داده های رایج علوم اعصاب محاسباتی وجود ندارد.
\footnote{اگرچه روشهای متعددی در تئوری اطلاعات وجود دارند که کمتر رایح هستند
	، اما از روش MI در این پروژه استفاده شده است.
}

\subsection{مکان یابی منابع سیگنال ها در مغز}

در اینجا مروری کوتاه بر مکانیابی سیگنال های مغزی انجام میدهیم، جزئیات این کار نسبتا مفصل است و در صورت نیاز به منابع مراجعه شود.
\footnote{\lr{https://pubmed.ncbi.nlm.nih.gov/24007117/}}
همانطور که تا کنون گفته شد، سیگنال مغزی بر هم نهی تعداد بسیار زیادی سیگنال الکتریکی حاصل از فعالیت نورون ها است. نورون ها را میتوان به صورت دو قطبی های الکتریکی که حامل جریان هستند مدل کرد، زیرا با فرض کوچک بودن همخوانی دارند. پس اگر بتوانیم تبدیل خطی معکوسی را که با اعمال بر روی این سیگنال ها، به سیگنال الکتریکی نورون ها می رسد را به دست بیاوریم، میتوان پاسخ تک تک نورون ها را در طول زمان به دست آورد. این کار عملا ممکن نیست؛ چرا که تعداد ثبت در حدود 20-60 سیگنال و تعداد نورون در حدود 
$10^11$
نورون است. بنا بر اساس معادلات خطی، تنها میتوانیم به تعداد معادلاتمان مجهول هارا به دست بیاوریم؛ بنابر این در صورت داشتن مختصات محل ثبت، میتوان تخمینی از 20-60 منبع به صورت مکانی در مغز به دست آورد. این روش مبتنی بر حل دستگاه معادلاتی به صورت زیر است:
$$ X(t) = G Q(t) $$
که در آن، X سیگنال ثبت شده و Q مشخصات دو قطبی منابع است. ماتریس G نیز مکان های ثبت سیگنال ها هستند؛ به دلیل بزرگ بودن این دستگاه معادلات پیاده سازی های نرم افزاری آن بر اساس روش های بهینه سازی انجام می شوند.

\begin{figure}[h!]
	\centering
	\includegraphics[width=16cm]{Images/9.JPG}
	\caption{فلوچارت مکان یابی منابع سیگنال ها بر اساس الکتروانسفالوگرام}
	\label{fig:14}
\end{figure}

\clearpage
\newpage

\section{محاسبه و فعالیت ذهنی در مغز}

در این بخش، نگاهی کلی به فیزیولوژی مغز در رابطه با فعالیت ذهنی و محاسبه می اندازیم. 

\subsection{فیزیولوژی فعالیت ذهنی}

فعالیت ذهنی و به خصوص محاسبه ذهنی، شامل در نظر گرفتن چند عدد در حافظه کوتاه مدت و انجام عملیات بر روی آنها، سپس قرار دادن حاصل در حافظه کوتاه مدت و در نهایت گزارش آن به صورت گفتاری، نوشتاری و یا به خاطر سپردن آن است. در تحقیقاتی که به خصوص در دو دهه اخیر انجام شده است؛ داده های fMRI  در حین انجام فعالیت های عددی و ذهنی به این نتیجه رسیده اند که تقریبا هر فعالیت مرتبط با اعداد در مغز، ناحیه های خاصی را بیشتر تحریک می کند. بدیهی است که هر فعالیت ذهنی ای فعالیت جانبی ممکن است داشته باشد؛ برای مثال حین ثبت سیگنال از افرادی که در حال محاسبه اند ممکن است فعالیت با مداد، کار با کامپیوتر و صفحه کلید، شنیدن و گفتن اعداد و نگاه به اعداد منجر به فعالیت نواحی گوناگونی از مغز خواهد شد که ممکن است در یافتن نواحی مرتبط تر ایجاد ابهام کند. برای رفع این مشکل، آزمایش ها و ثبت ها به به اکثر روش های ممکن تکرار می شوند، سپس روش هایی مانند میانگین گیری به ما کمک میکنند تا ناحیه ای که فقط در حالات خاصی مانند محاسبه با کاغذ، بر روی صفحه نمایشگر و یا کارهای شنیداری فعال است یافته نشود.

\begin{figure}[h!]
	\centering
	\includegraphics[width=13cm, height=8cm]{Images/7.JPG}
	\caption{\lr{Insula, MTG and STG}}
	\label{fig:15}
\end{figure}

\subsection{نواحی مرتبط با محاسبه ذهنی}

 بر اساس ثبت های مختلف سیگنال ها حین این فعالیت های عددی، بیشترین فعالیت ثبت شده به صورت میانگین در همه سابجکت ها، نواحی برآمدگی میانی گیجگاهی(MTG)
\footnote{\lr{Mid-Temporal Gyrus}}
و قشر اینسولا (INS)
\footnote{\lr{Insula Cortex}}
هستند. ناحیه MTG که به نواحی مرتبط با حافظه کوتاه مدت مربوط است و ناحیه INS نیز تقریبا در همه فعالیت های مربوط به اعداد در تقریبا همه حالات فعال است؛ مهمترین اهمیت این اطلاعات در باره فعالیت این نواحی این است که میتوان با یافتن کانال های نزدیک تر و مرتبط تر به این نواحی در سیگنال های الکتروانسفالوگرام، دقت طبقه بند ها و خوشه بند ها را بهبود داد. نواحی مرتبط دیگر که برای افراد مختلف و فعالیت های مختلف هر از گاهی متفاوت است؛ قشر چنگی میانی (MCC)
\footnote{\lr{Mid-Cingulate Cortex}}
و لوب آهیانه ای پایینی (IPL)
\footnote{\lr{Inf-Parietal Lobe}}
 و برآمدگی پایینی گیجگاهی (STG)
\footnote{\lr{Sub-Temporal Gyrus}}
  هستند.

\subsection{ویژگی های محاسبه ذهنی در سیگنال های مغزی}

هر چند در پردازش های مرتبط با این فعالیت ها تمامی جنبه ها و ویژگی های سیگنال جهت اطمینان بیشتر در نظر گرفته می شوند، اما برخی ویژگی ها انتظار می رود بیشتر موثر باشند. بر اساس ویژگی های فرکانسی مغز، بسیاری از فعالیت های مغز را میتوان تشخیص داد؛ برای مثال فرکانس های پایین در سیگنال های الکتروانسفالوگرام در صورت داشتن الگو های خاصی بیانگر ناهنجاری هایی میتوانند باشند و یا فعالیت های فرکانس خاصی تنها در حالت هوشیاری کامل ممکن هستند. این دانسته های بسیار ارزشمند باعث میشوند بسیاری از بخش های سیگنال که ضریب اهمیت کمتری در طبقه بندی سیگنال ها دارند در مرحله پیش پردازش حذف شده و تصمیم گیری ها درباره سیگنال و ویژگی های اصلی ساده تر شوند. 

\begin{figure}[hb!]
	\centering
	\includegraphics[width=12cm, height=6cm]{Images/6.png}
	\caption{باند های فرکانسی و فعالیت ها}
	\label{fig:16}
\end{figure}

ویژگی مهم فعالیت های محاسباتی ذهنی، این است که در رده فعالیت های فکری سطح بالا قرار میگیرند؛ اکثر این فعالیت ها در فرکانس های بالای 15 هرتز مانند باند بتا و گاما در سیگنال های الکتروانسفالوگرام مشاهده می شوند. نتایج طبقه بندی با در نظر گرفتن و نگرفتن فرکانس های دیگر نشان می دهد که این ادعا صحت دارد؛ و در حال حاضر نیز برای بررسی فعالیت های ذهنی نیازمند هوشباری بالا، فرکانس های بالای 15 هرتز فقط در نظر گرفته می شوند. 


\clearpage
\newpage

\section{خوشه بندی و طبقه بندی سیگنال های مغزی در طول زمان}

یکی از اهداف مطالعه سیگنال های مغزی، تشخیص رفتار های گوناگون ذهنی بر اساس سیگنال های مغزی است. این کار بدون وجود قابلیت جدایی پذیری سیگنال ها بر اساس برچسپ خاص ممکن نیست؛ تصور کنید ویژگی ای وجود نداشته باشد که بتوان یک سیگنال که حین خواب ثبت شده را از یک سبگنال در حالت بیداری تمییز داد و هر دو سیگنال کاملا مشابه باشند؛ در آن صورت هر تحقیقی بر روی این سیگنال ها بی نتیجه می بود. به کمک ابزار های طبقه بندی، خوشه بندی و جداسازی مولفه های سیگنال ها می توان بررسی نمود آیا یک فعالیت خاص موجب رفتار یا پاسخ خاصی در یک سیستم می شود یا آن فعالیت از طریق این روش قابل مطالعه نیست.

\subsection{جدا سازی منابع سیگنال های مغزی}

تصور کنید که در محیط یک رستوران در حال شنیدن صداهای مختلف هستید؛ صدای ظرف ها، صدای صحبت کردن مردم، صدای موسیقی زمینه و صدای خیابان همزمان به گوش شما می رسند. هر صدا، منبع و عمل جداگانه ای دارد و در صورتی که بدانید منبع آن چیست، عامل آن را نیز احتمالا شناسایی کرده اید. حال اگر سیگنال های ثبت شده از مغز حین فعالیت خاص را بررسی کنیم، عوامل مختلفی این سیگنال را ایجاد کرده اند که میتوانند به علت عامل خاصی مانند تحریک بینایی باشند. در صورتی که متوجه شویم تحریک بینایی سبب تحریک منبع خاصی به تولید سیگنال خاصی شده است، میتوان گفت آن منبع پاسخ مغز به تحریک است.

\subsubsection{آنالیز مولفه های اصلی}

آنالیز مولفه های اصلی	
\footnote{\lr{Principle component analysis (PCA)}}
به طور کلی، یک دسته داده را به مولفه هایی تجزیه می کند که بر اساس مقادیر ویژه ماتریس کواریانس و بردار ویژه متناظر آنها جدا شده اند. این تبدیل بر روی ماتریس معین مثبت یا نیمه معین مثبت قابل انجام است؛ به همین دلیل لازم است داده ها به طریقی به یک چنین ماتریسی تبدیل شوند؛ بنا بر خاصیت ماتریس کواریانس که همواره نیمه معین یا معین مثبت است برای آنالیز مولفه های اصلی از آن استفاده می شود. به طور کلی فرم ریاضی به صورت زیر است:

$$ For \: signal \: x(t) \rightarrow \: C_x = E[(x-\bar{x})(x-\bar{x})], \: C_x u_i = \lambda_i u_i \: \rightarrow C_x U = \Lambda U \rightarrow U^{-1} = U^{T} \: (Orthogonality)$$
$$ C_x = U \Lambda U^{T}, U^{T} C_x U = \Lambda $$

در نهایت؛ این روش بر اساس تجزیه مقادیر ویژه ماتریس کواریانس ترکیب خطی ای به ما میدهد که با اعمال بر روی دسته سیگنال ها، تخمینی از منابع اصلی میدهد. این روش از اطلاعات پیشین
\footnote{\lr{Prior knowledge}}
 استفاده نمیکند و یک روش کور
\footnote{\lr{Blind}}
  محسوب می شود. در این پروژه بیشتر به پیاده سازی نرم افزاری این روش بسنده شده است که بر مبنای الگوریتم های تکرار 
\footnote{\lr{Iterative}}
است.

\subsubsection{آنالیز مولفه های مستقل}
	
آنالیز مولفه های مستقل
\footnote{\lr{Independant component analysis (ICA)}}
هدفی مشابه PCA دارد؛ با این تفاوت که مولفه ها مستقل آماری باشند. فرض استقلال بر روی منابع یک مزیت مهم دارد؛ و آن این است که اگر یک عملی موجب ایجاد یک سیگنال خاص شده است، این سیگنال باید مستقل از سایر منابع و فقط به این عمل وابسته باشد. فرم کلی ریاضی این روش به صورت زیر است:

$$ x(t) = A s(t), A^{-1} x(t) = s(t), W = A^{-1} =\rightarrow s(t) = W x(t) $$

هدف تخمین W است به گونه ای که منابع s مستقل آماری باشند؛ بنابر قضایای آماری مانند حد مرکزی و اطلاعات مشترک، با حل مساله مینیمم سازی اطلاعات مشترک و ماکسیمم ناگوسی بودن که بر اساس معیار هایی که با کومولان ها تعریف می شوند میتوان تا حد امکان منابعی مستقل تخمین زد. این الگوریتم نیز در پیاده سازی بر مبنای تکرار در اکثر کتابخانه های آماری پیاده شده است.

\subsection{طبقه بندی}

در این بخش، بحث طبقه بندی سیگنال ها بین دو دسته کیفیت بالا و تعداد بالای محاسبه و دسته کیفیت پایین و تعداد پایین محاسبه را بررسی میکنیم. جهت طبقه بندی، از یک شبکه عصبی چند لایه 
(\lr{Fully connected network (FCN)})
استفاده کرده ایم؛ 

\subsubsection{طبقه بندی بر اساس تعداد محاسبه}

هدف از این طبقه بندی، بررسی فرایند محاسبه در سیگنال های مختلف است. در داده تست، یک دسته افراد تعداد محاسبه بالا در حدود 25 در دقیقه و یک دسته تعداد محاسبه پایین در حدود 10 در دقیقه داشتند؛ برای مشاهده تفاوت های احتمالی سیگنال مغزی آنها از یک طبقه بند کانولوشنال عصبی استفاده شد تا در مرحله مقدماتی شباهت های ممکن بررسی گردد. مشاهده شد که برای افراد دارای تعداد محاسبه بیشتر، فرکانس خود همبستگی سیگنال بیشتر است و با تعداد محاسبه ارتباط دارد؛ اگرچه این ارتباط دارای خطا است ولی برای یک فرد با 4 محاسبه و یک فرد دیگر با 27 محاسبه به وضوح این تفاوت دیده میشود:

\begin{figure}[h!]
	\centering
	\includegraphics[width=16cm]{Images/8.JPG}
	\caption{خود همبستگی تکه ای از سیگنال برای دو فرد مختلف}
	\label{fig:17}
\end{figure}

\subsubsection{طبقه بندی بر اساس کیفیت و تعداد محاسبه}

در داده اولیه، برچسپی به عنوان کیفیت محاسبه هست که در آن؛ فردی که بالای 80 درصد محاسباتش درست است خوب و بقیه بد در نظر گرفته شده اند. این به ما کمک می کند معیاری از ویژگی سیگنال فرد با تمرکز بهتر داشته باشیم تا گراف احتمالی دارای خطایی از نوع عدم تمرکز بر فعالیت نداشته باشد تا فرایند تخمین زده شده از محاسبه ذهنی دارای خطا شود. 

\subsection{خوشه بندی}

تمرکز بیشتر این پروژه، از این بخش آغاز می شود. در قسمت قبلی گفتیم که امکان جدا شدن سیگنال ها بر اساس ویژگی های مختلف، یک خاصیت خوب برای آن سیگنال ها است و با انجام طبقه بندی بر روی داده ها و گرفتن یک دقت بالای 50 درصد که نشان از تصادفی نبودن این نتیجه را با تکرار آن نشان دادیم. در این زیر بخش کلیت روش هایی برای بررسی این امکان به صورت بدون سرپرست و مستقل از برچسپ ها را مرور میکنیم و نتایج خوشه بندی بر روی این سیگنال ها بر اساس گراف هایی که طبق روش های مختلف استخراج شده اند را بررسی و تحلیل میکنیم. 

\subsubsection{بررسی جدایی پذیری و شباهت سیگنال ها بین افراد}

به طور کلی الگوریتم های خوشه بندی سعی دارند بر مبنای تبدیل های ریاضی، داده ها را به یک فضای دوم نگاشت می کنند؛ سپس در این نگاشت در صورتی که داده ها به صورت خوشه های جداگانه قرار گیرند، میتوان گفت که آنها در دسته هایی مجزا قرار می گیرند که هر اعضای هر دسته به هم ارتباط دارند. ساده ترین روش خوشه بندی، روش میانگین چندتایی
\footnote{\lr{K-means}}
است که در آن، به صورت تکرار شونده
\footnote{\lr{Iterative}}
هر یک سری مراکز در نظر گرفته می شود و با ادغام آنها، میانگین هر دسته به عنوان مرکز جدید در نظر گرفته می شود. فرمول کلی ای برای این روش نمیتوان بیان کرد چرا که بسته به فرم داده ها برای هر نوعی متفاوت میشود. در این پروژه بیشتر از دو روش PCA و TSNE استفاده شده است که در ادامه آنها را توضیح میدهیم.

\subsubsection{خوشه بندی بر اساس K-means}

\subsubsection{خوشه بندی بر اساس PCA}

\subsubsection{خوشه بندی بر اساس TSNE}

\subsubsection{خوشه بندی ها بر روی گراف های مبتنی بر همبستگی طیفی}

\subsubsection{خوشه بندی ها بر روی گراف های مبتنی بر اطلاعات مشترک}

\subsubsection{خوشه بندی ها بر روی گراف های مبتنی بر تابع علیت گرنجر}

\subsubsection{مقایسه نتایج روش ها}

\clearpage
\newpage

\section{جمع بندی}

کلیت پروژه و شرحی از اهداف آن تا کنون گفته شد، در این زیر بخش برنامه زمانی کلی و هدف گذاری ای تقریبی از ادامه این پروژه در نیمسال آینده شرح خواهیم داد. 
\subsection{بحث و بررسی}

\subsection{خلاصه نتایج}

\subsection{در ادامه}


\clearpage
\newpage
\appendix

\section{پیوست}

\subsection{شکل و نمودار ها}

\listoffigures

\newpage
\subsection{منابع}

\lr{
\begin{itemize}
	\item[1]
	Analysis of functional connectivity based on EEGs for mental state classification [S. Qodsi et. al, 2019]
	\item[2]
	EEG during mental arithmetic task performance [I. Zyma et. al, 2017]
	\item[3]
	Brain areas associated with numbers and calculations[M.Arsalidou et. al, 2017]
	\item[4]
	Age related differences in the structural and effective connectivity of cognitive control [T. Hinault et. al, 2019]
	\item[5]
	Dynamic causal modeling for EEG and MEG [S. J. Kiebel et. al, 2008]
\end{itemize}
}

\subsection{منابع پیاده سازی و داده ها}

\lr{
\begin{itemize}
	\item[1]
	MNE-Python library [https://mne.tools/stable/index.html]
	\item[2]
	PyMVPA-Python library [http://www.pymvpa.org]
	\item[3]
	EEG During Mental Arithmetic Tasks[https://www.physionet.org/content/eegmat/1.0.0]
	\item[4]
	Tensorflow library [https://www.tensorflow.org]
\end{itemize}
}

\end{document}